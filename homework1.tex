\documentclass[12pt, oneside]{article}   	% use "amsart" instead of "article" for AMSLaTeX format
\usepackage[utf8]{inputenc}
\usepackage{geometry}                		% See geometry.pdf to learn the layout options. There are lots.

\usepackage{libertine}
\usepackage{setspace}
\usepackage{mathtools}
                  		% ... or a4paper or a5paper or ... 
%\geometry{landscape}                		% Activate for rotated page geometry
%\usepackage[parfill]{parskip}    
\usepackage{graphicx}				% Use pdf, png, jpg, or eps§ with pdflatex; use eps in DVI mode		

\usepackage{amssymb}
\usepackage{epsfig}
\usepackage{url}

%header and footer
\usepackage{fancyhdr}
\pagestyle{fancy}
\usepackage{datetime}

\begin{document}
	
\begin{titlepage}
	\begin{center}
	\line(1,0){400} \\
    [0.25in]
    \Huge{\bfseries CS 210, Spring- Homwork 1} \\
    [2mm]
    \line(1,0){400} \\
    [3 cm]
    
    \textsc{\LARGE Kamalika Poddar} \\
   
  
    \textsc{\LARGE University of California, Riverside} \\
    [0.7cm]
  \vspace*{7 cm}
  \today
    \end{center} 

\end{titlepage}

\newpage
\vspace{2cm}

\begin{center}
\textbf{ \Large ANSWERS}\\
\end{center}

\vspace{0.5cm}
\textbf{ \Large Errors and sources of error}\\

\begin{enumerate}
	\item \textbf{ What are the approximate absolute and relative errors in approximating Euler’s number e by each of the following quantities? }\\
	
	Absolute Error(Ae) = approximate value - true value\\
	Relative error(Re) = absolute error/true value\\
	Approximate Euler's number= 2.71828
	\begin{enumerate}
		\item \textbf{3:}\\
		Ae= 3-2.71828=0.28171\\
		Re= 0.28171/2.71828= 0.10363893 $\approx 10 \%$
		\item \textbf{2.17:}\\
		Ae= 2.71-2.71828= - 0.00828 \\
		Re= - 8.28 x $10^{-3}$/2.71828= -0.0030460$\approx -0.3 \%$
		\item \textbf{19/7:} This is equal to 2.714285\\
		Ae= 2.714285-2.71828=- 0.0039942\\
		Re= - 0.0039942/2.71828= - 0.0014694$\approx -0.1 \%$
	\end{enumerate}

\hspace{3cm}
	\item \textbf{For each of the following statements, indicate whether the statement is true or false.}
	\begin{enumerate}
		\item \textbf{FALSE} -
		A large absolute error implies a large relative error. 
		\item \textbf{TRUE}-  Truncation error is due to approximations such as truncating an infinite series while rounding error is due to inexactness in the representation of real numbers. 
		\item \textbf{TRUE}- Computational error consists of the difference between the exact and the approximated functions calculated on the same inexact input data.
		\item \textbf{FALSE}- \^f(x) = x can be used to approximate f(x) = sin(x) for large values of x.
	\end{enumerate}
\newpage
\textbf{ \Large Conditioninig and stability}\\
      \item \textbf{For each of the following statements, indicate whether the statement is true or false.}
      	\begin{enumerate}
      	\item \textbf{FALSE}
      	Using higher-precision arithmetic will make an ill-conditioned problem better conditioned.  
      	\item \textbf{TRUE} A stable algorithm applied to a well-conditioned problem necessarily produces an accurate solution. 
      	\item \textbf{FALSE} - A condition number of 1 means the problem is ill-conditioned.   
      	
      \end{enumerate}
  
  \hspace{3cm}
    	\item \textbf{Consider the function $f:R -> R$ defined by $f(x,y)= x-y$. Measuring the size of the input $(x,y)$ by $|x| + |y|$, and assuming that $|x| + |y| \approx 1 $ and $x-y \approx \epsilon$, show that $cond(f) \approx 1/ \epsilon$. What can you conclude about the sensitivity of substraction?}\\
    	
    	
    	
    	Condition number$= \frac{|(f(\hat {x})- f(x))/ f(x)| }{|(\hat{x }-x )/x|}$ $= \frac{|(\hat{y}- y)/ y| }{|(\hat {x} -x )/x|}$$= \frac{|(\Delta y)/ y| }{|(\Delta x )/x|}$ \\
    	
    	Condition number= $\frac{ |Relative forward error| }{|Relative backward error|}$\\
    	
    	When a function is evaluated by a differential function : $ f:R ->R$, the condition number so obtained by substituting relative forward and backward error on two variable x and y  is:\\
    	
    	Condition number $\approx |\frac{f(x + \Delta x, y + \Delta y) - f(x,y) / f(x,y)} { |\Delta x| + | \Delta y|/|x| + |y|}|$\\
    	
    	Substituting $f(x,y)= x-y$ , $|x| + |y| \approx 1 $ and $x-y \approx \epsilon$\\
    	
    	Condition number $\approx |\frac{f(x + \Delta x, y + \Delta y) - \epsilon / \epsilon} { |\Delta x| + | \Delta y|/1}|$\\
    	
    \hspace{3cm}	$\approx |\frac{x + \Delta x -y - \Delta y - \epsilon / \epsilon} { |\Delta x| + | \Delta y|/1}|$\\
    	
     \hspace{3cm}	$\approx |\frac{ \Delta x  - \Delta y  / \epsilon} { |\Delta x| + | \Delta y|/1}|$\\
     
       \hspace{3cm}	$\approx |\frac{1}{\epsilon}|\frac{ |\Delta x  - \Delta y| } { |\Delta x| + | \Delta y|}$\\
       
       So for absolute number, we have\\
       
         \hspace{3cm}	$\approx |\frac{1}{\epsilon}|\frac{ |\Delta x | +| \Delta y| } { |\Delta x| + | \Delta y|} = \frac{1}{\epsilon}$\\
         
         Therefore we could prove that the condition number of the function is $\frac{1}{\epsilon}$.\\
         We conclude that the sensitivity of substraction is more as the distance between the operands decreases.\\
         
         \hspace{4cm}
         
         
    	
    	
\textbf{ \Large Floating point}\\	
		\item \textbf{A floating point number system is characterized by four integers: the base $\beta$, the precision p, and the lower and upper limits L and U of the exponent range.}
		\begin{enumerate}
		\item  If $\beta$ = 10, what are the smallest values of p and U, and the largest value of L, such that both 12.3456 and 0.0012 can be represented exactly in a normalized floating-point system?\\
		 
		 Numbers represented in a normalized system are:\\
		 $12.3456= 1.23456 * 10^1$\\
		 $0.0012= 1.2 * 10^{-3}$\\
		 
		 Smallest value of precision (p) is = 6\\
		 Lower limit(L)= -3\\
		 Upper limit(U)= 1\\
		 
		   \hspace{3cm}
		\item How would your answer change if gradual underflow is allowed?\\
		
		If we denormalize 0.0012, representing six digits of precision, we have $0.00012 *10^1$, L=1, U= 1\\
		Further increasing Precision to 7, we have $0.000012*10^2$, L=1 and U=2.  
		So with the increase in precision there is a increase in U.
	\end{enumerate}

 \hspace{3cm}
	\item \textbf{Between an adjacent pair of nonzero IEEE single precision real numbers, how many IEEE doule precision numbers are there?}\\
	
	In a floating point number the next adjacent number obtained will the increment of its least significant bit by 1.
	 For single precision number we have precision as 24 bits and two adjacent numbers are represented as- \\
	 $1.23.......3330  * 2^e$ and $1.23.......3331 *2^e$ , there are 23 bits after the leftmost bit(1).\\
	 
	 For a double precision number there are 53 bits precision but between two adjacent number we have 52 bits precision.\\
	 A double precision number has 29 additional bits for mantissa more  than single precision bit.
	 Therefore between two adjacent pair of a single precision number, we have 29 less significant bits of a double precision number.\\
	 The total number of double precision numbers  excluding the case when all the additional numbers are zero:\\ $2^{29} -1 = 536870911$.
	
	 \hspace{3cm}
	\item \textbf{For each of the following statements, indicate whether the statement is true or false.}
	\begin{enumerate}
		\item \textbf{FALSE}-
		If two real numbers are exactly represented as floating point numbers, then the result of a arithmetic operation on them will also be represented as a floating point number.
		\item \textbf{FALSE}- Floating-point numbers are distributed uniformly throughout their range.
		\item \textbf{TRUE}- Floating point arithmetic is commutative, but not associative.
		\item \textbf{FALSE}- The denormalized floating point representation of a number is unique.
		\item \textbf{FALSE}- Addition of two positive floating point numbers may cause underflow.
		\item \textbf{TRUE}- Division of two positive floating point numbers may cause overflow.
		\item \textbf{FALSE} -In a normalized floating point system, the representation of a machine number is not unique.
		
	\end{enumerate}
	
	\newpage
	
	\item \textbf{ a.} Given:\\
	$V= 0.5m^3$,  $R= 8.31 J mol^{-1} K^{-1}$\\
	$PV=nRT \implies n=\frac{PV}{RT}$\\
	
	Relative forward error for Pressure= $\frac{f(x+\Delta x ) -f(x)} { f(x)}$   \\ 
	
	\hspace{6cm}= $|\frac{\frac{(P+\Delta P)V }{RT}-\frac{(PV) }{RT}} { \frac{(PV) }{RT}}|$  \\
	
		\hspace{6cm}= $|\frac{\frac{(PV+\Delta PV -PV }{RT} } { \frac{(PV) }{RT}}|$  \\
		
		\hspace{6cm}=$\frac{\Delta P}{P}= \epsilon$\\
			
		Relative forward error for Temperature= $\frac{f(x+\Delta x ) -f(x)} { f(x)}$   \\ 
			
		\hspace{7cm}= $|\frac{\frac{PV }{R(T+ \Delta T)}-\frac{(PV) }{RT}} { \frac{(PV) }{RT}}|$  \\
			
		\hspace{7cm}= $|\frac{\frac{PVRT -PVRT - PVR\Delta T}{R*RT(T+ \Delta T)}} { \frac{(PV) }{RT}}|$  \\
		
		\hspace{7cm}= $|\frac{\frac{- PVR\Delta T}{R*RT(T+ \Delta T)}} { \frac{(PV) }{RT}}|$  \\
				
		\hspace{7cm}=$\frac{\Delta T}{T + \Delta T}$\\
		
		As T is small, $\Delta T$ will be very very small. Therefore error equals to :
		$ \frac{\Delta T}{T }= \epsilon $\\
		Total relative forward error for the amount of gas is $\frac{\Delta n}{n} $= $\epsilon + \epsilon=2 \epsilon$\\
		
		\textbf{b.}   To find the absolute and relative errors:\\
			$	n= PV/RT=100 *0.5/ 8.31 *300 = \textbf{0.0200}$ moles\\	
		\begin{itemize}
			\item    	$\Delta n$= $\frac{(P+ \epsilon P) V}{R (T+ \epsilon T)}= 101* 0.5 / 8.31 * 300.5=\textbf{0.02022 moles}$\\
			
			Absolute error: ${0.02022- 0.0200}= 2.2*10^{-4} $\\
			
			Relative error: $\frac{0.02022- 0.0200}{0.0200} = 0.011 \approx \textbf{1.1\%}$
			
			\item  $\Delta n$= $\frac{(P-\epsilon P) V}{R (T-\epsilon T)}= 99* 0.5 / 8.31 * 299.5=\textbf{0.0198887 moles}$\\
			
			Absolute error: ${0.0198887- 0.0200}= -3.31256*10^{-4} $\\
			
			Relative error: $\frac{-3.31256*10^{-4} }{0.0200} = -0.01656 \approx \textbf{-1.6\%}$
			
			\item  $\Delta n$= $\frac{(P+ \epsilon P) V}{R (T- \epsilon T)}= 101* 0.5 / 8.31 * 299.5=\textbf{0.02029 moles}$\\
			
			Absolute error: ${0.02029 - 0.0200}= 2.905*10^{-4} $\\
			
			Relative error: $\frac{2.905*10^{-4}}{0.0200} = 0.0145 \approx \textbf{1.4\%}$
			
			\item  $\Delta n$= $\frac{(P-\epsilon P) V}{R (T+\epsilon T)}= 99* 0.5 / 8.31 * 300.5=\textbf{0.0198225 moles}$\\
			
			Absolute error: ${0.0198225- 0.0200}= -1.7744*10^{-4} $\\
			
			Relative error: $\frac{-1.7744*10^{-4}}{0.0200} = -0.00872 \approx \textbf{0.8\%}$\\
			
			\textbf{The worst absolute and relative error is obtained when $P=P+ \epsilon P$ and $T= T-\epsilon T$.}
			
		\end{itemize}
	
		
		\item Below table shows the obtained Gaussian functions. The programming language I used to perform the calculation is Python and I have attached the code with the report. \\
		
		 \begin{center}
			\begin{tabular}{ c|c| c| c }
				&$\mu=0$ & $\mu=1$& $\mu=2$ \\ 
				\hline
				$\sigma= 0.01$ & 4.7463 & 0.000& 0.00\\  
				\hline
					$\sigma= 0.10$ & 0.8110 & 1.025 & 7.36 \\  
				\hline
					$\sigma= 1.00$ & 1.00 & 0.999 & 0.999 \\  
				\hline  
					$\sigma= 10.0$ & 1.00 &1.000 &1.000 \\  
				\hline  
			\end{tabular}
		\end{center}
	The obtained results are shown as:
	\begin{flushleft}
			\includegraphics[width=15cm, height=15cm]{gaussian}
	\end{flushleft}

	Multivariate version of Gaussian distribution results are obtained manually:\\
	 \begin{center}
		\begin{tabular}{ c|c| c }
			&$k=2$ & $k=3$ \\ 
			\hline
			$\sigma= 0.1$ & $3.3911* 10^{-3}$& $1.3536 * 10^{-3}$\\  
			\hline
			$\sigma= 1.0$ & 0.09643 & 0.03850\\  
			\hline
			$\sigma= 2.0$ &  0.087560& 0.03496 \\  
			\hline  
		 
		\end{tabular}
	\end{center}

\end{enumerate}




\vspace{1cm}
\textbf{ \Large References}\\
\begin{enumerate}
	\item " Scientific Computing"- An Introductory Survey by Michael T. Health.
	\item Classnotes from CS-210
	\item https://docs.scipy.org/doc/scipy/reference/generated/scipy.integrate.romberg.html
	\#scipy.integrate.romberg
	\item http://lagrange.univ-lyon1.fr/docs/scipy/0.17.1/generated/scipy.stats.multivariate\_
	normal.html
	\item https://stackoverflow.com/questions/14873203/plotting-of-1-dimensional-gaussian-distribution-function
\end{enumerate}
\end{document}
